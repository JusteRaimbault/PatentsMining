%%%%%%%%%%%%%%%%%%%%%%%%%%%%%
% Standard header for working papers
%
% WPHeader.tex
%
%%%%%%%%%%%%%%%%%%%%%%%%%%%%%

\documentclass[11pt]{article}

%%%%%%%%%%%%%%%%%%%%
%% Include general header where common packages are defined
%%%%%%%%%%%%%%%%%%%%



%%%%%%%%%%%%%%%%%%%%%%%%%%
%% Packages
%%%%%%%%%%%%%%%%%%%%%%%%%%


% general packages without options
\usepackage{amsmath,amssymb,bbm}

% graphics
\usepackage{graphicx}

% text formatting
\usepackage[document]{ragged2e}
\usepackage{pagecolor,color}



%%%%%%%%%%%%%%%%%%%%
%% Idem general commands
%%%%%%%%%%%%%%%%%%%%
%% Commands

\newcommand{\noun}[1]{\textsc{#1}}


%% Math

% Operators
\DeclareMathOperator{\Cov}{Cov}
\DeclareMathOperator{\Var}{Var}
\DeclareMathOperator{\E}{\mathbb{E}}
\DeclareMathOperator{\Proba}{\mathbb{P}}

\newcommand{\Covb}[2]{\ensuremath{\Cov\!\left[#1,#2\right]}}
\newcommand{\Eb}[1]{\ensuremath{\E\!\left[#1\right]}}
\newcommand{\Pb}[1]{\ensuremath{\Proba\!\left[#1\right]}}
\newcommand{\Varb}[1]{\ensuremath{\Var\!\left[#1\right]}}

% norm
\newcommand{\norm}[1]{\| #1 \|}


%% graphics

% renew graphics command for relative path providment only ?
%\renewcommand{\includegraphics[]{}}



% geometry
\usepackage[margin=2cm]{geometry}

% layout : use fancyhdr package
\usepackage{fancyhdr}
\pagestyle{fancy}

\makeatletter

\renewcommand{\headrulewidth}{0.4pt}
\renewcommand{\footrulewidth}{0.4pt}
\fancyhead[RO,RE]{\textit{Working Paper}}
\fancyhead[LO,LE]{G{\'e}ographie-Cit{\'e}s/LVMT}
\fancyfoot[RO,RE] {\thepage}
\fancyfoot[LO,LE] {\noun{J. Raimbault}}
\fancyfoot[CO,CE] {}

\makeatother


%%%%%%%%%%%%%%%%%%%%%
%% Begin doc
%%%%%%%%%%%%%%%%%%%%%

\begin{document}






\title{Mining Patents Data through Semantic Network Analysis\bigskip\\
\textit{Project Proposal}
}
\author{\noun{J. Raimbault}}
\date{September 19th}


\maketitle

\justify


\begin{abstract}
Patents are a central proxy in the study of the economy of innovation. Indeed, the information contained in relations between patents reflects the underlying structure of the socio-technological system of research and development in innovative companies. Whereas recent focus was mainly on the study of patterns in the inter-patent citation network, taking technological fields as externally fixed, we propose a novel approach based on semantic analysis of patents textual contents. Indeed, measures such as co-occurences and repetitions of keywords should contain a slightly different information than the one extracted from the citation network, as for example links between domains, or informal domains that could appear as communities in the semantic network. The aim of this project is to investigate the nature and extent of this information, by the mining of regular patterns in features that should be established. We expect to test various features crossing measures extracted from both dynamic citation and semantic network, by unsupervised, and supervised if needed, datamining techniques. Expected results are to unveil the potentialities of such approach, and in the case of significant information, to apply it to classify patents and be able to systematically differentiate innovatives from imitating patents.
\end{abstract}


%%%%%%%%%%%%%%%%%%%%
\section{Introduction}

% state-of-the-art and recent works.


The study of innovation through the lens of technological patents is not a novel idea~\cite{basberg1987patents} but the recent rise of new methods and computational abilities, including datamining and network analysis~\cite{newman2010networks} has shed a new light on the approach. With methods relatively close to applied epistemology studies such as citation dynamics modeling~\cite{2013arXiv1310.8220N} or co-autorships networks analysis~\cite{2014arXiv1402.7268S}, recent works have studied patents citation network to understand the processes of technological innovation.






%%%%%%%%%%%%%%%%%%%%
\section{Research objectives}


\subsection{Research question}

The identification of so-called \emph{emerging research fronts} was done in the case of scientific publications in~\cite{shibata2008detecting}. In the same spirit, our guiding research objective is to identify such fronts for patents, i.e. to be able to classify patents in order to distinguish the ``real innovations'' from imitations of existing technologies. That is relatively close to the issue of defining a relevant measure of innovation~\cite{archibugi1988search}


% cross domains ? -> Youn paper. \cite{Youn:2015fk}


\subsection{Link with previous work}

% particular state-of-the-art ; how are we positioned ?

A consequent amount of research already proposed to use semantic networks to study technological domains. One of the first works to enhance the approach was~\cite{yoon2004text}, where the idea of visualizing keywords network was introduced and illustrated on a small technological domain.

The novelty of our project relies on various points, including the systematic unsupervized research of patterns, the possibility to build hybrid features from both semantic and citation networks, the combination of various techniques from datamining to network analysis.



%%%%%%%%%%%%%%%%%%%%
\section{Proposed approach}

%%%%%%%%%%%%%%%%%%%%
\subsection{General description}

This project is based on the assumption that semantic relations between patents contents must include some information on the underlying technological innovation processes. The idea is strongly inspired from the semantic science mapping proposed in~\cite{chavalarias2013phylomemetic}, where dynamics of scientific fields was reconstructed by keywords co-occurence networks analysis. The assets of such an approach include

Every patent is associated with many details that have been made available by national and international patent offices. First, each patent is associated with one or several technological fields which are chosen from a list by patent office specialists after careful review of the content of the patent (see IPC class). Second, every patent has to make a list of citations to all the patents used in the process, this naturally define a network between entities that have some technologies in common. From these two features innovation economists have created various indices to measure the quality of a patent ( number of citations, scope of technological field...).

Two other indicators are of particular interest for us: originality and radicalness. The originality measure is
defined by Hall \textit{et al.} (2001) \cite{Hall2001} as $Ori_{i}%
=1-\displaystyle\sum_{j=1}^{n_{i}}{cit_{i,j}^{2}}$ where $cit_{i,j}$ is the
percentage of citations made by patent $i$ to a patent in class $j$ out of
$n_{i}$ technological classes to which patent $i$ belongs. If the scope of
technologies which the patent uses and cites is large, then the originality
measure will be high. Radicalness is more difficult to define. It is
constructed in the same way as the originality index but here we only consider
the technology classes of patents cited by patent $i$ but to which patent $i$
does not belong.


%%%%%%%%%%%%%%%%%%%%
\subsection{Technical details}






%%%%%%%%%%%%%%%%%%%%
\section{Project organisation}


\begin{itemize}
\item Project definition, litterature review - AB, JR, PA - ETA 10h
\item Thematic framing (possible features, precise objectives) - AB, JR - ETA 4h
\item Technical aspects (database management, text-mining implementation) - JR - ETA 10h
\item Empirical datamining - AB, JR - ETA 20h
\item Theoretical feedback, revision of mining techniques - AB, JR, PA - ETA 10h
\item Theoretical modeling, results interpretation, perspectives - AB, PA - ETA ?
\end{itemize}













%%%%%%%%%%%%%%%%%%%%
%% Biblio
%%%%%%%%%%%%%%%%%%%%

\bibliographystyle{apalike}
\bibliography{../../Biblio/patents}


\end{document}


