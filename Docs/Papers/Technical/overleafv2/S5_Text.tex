\documentclass[10pt,A4]{article}
\usepackage[top=0.85in,left=2.75in,footskip=0.75in]{geometry}

% Use adjustwidth environment to exceed column width (see example table in text)
\usepackage{changepage}

% Use Unicode characters when possible
\usepackage[utf8]{inputenc}
\usepackage{booktabs,caption,fixltx2e}
\usepackage[flushleft]{threeparttable}
\usepackage{tabularx}
% textcomp package and marvosym package for additional characters
\usepackage{textcomp,marvosym}

% fixltx2e package for \textsubscript
\usepackage{fixltx2e}

% amsmath and amssymb packages, useful for mathematical formulas and symbols
\usepackage{amsmath,amssymb}

% cite package, to clean up citations in the main text. Do not remove.
\usepackage{cite}

% Use nameref to cite supporting information files (see Supporting Information section for more info)
\usepackage{nameref,hyperref}

% line numbers
\usepackage[right]{lineno}

% ligatures disabled
\usepackage{microtype}
%\DisableLigatures[f]{encoding = *, family = * }

% rotating package for sideways tables
\usepackage{rotating}

% Remove comment for double spacing
\usepackage{setspace} 
\doublespacing

% Text layout
\raggedright
\setlength{\parindent}{0.5cm}
\textwidth 5.25in 
\textheight 8.75in

% Bold the 'Figure #' in the caption and separate it from the title/caption with a period
% Captions will be left justified
\usepackage[aboveskip=1pt,labelfont=bf,labelsep=period,justification=raggedright,singlelinecheck=off]{caption}

% Use the PLoS provided BiBTeX style
\bibliographystyle{plos2015}

% Remove brackets from numbering in List of References
\makeatletter
\renewcommand{\@biblabel}[1]{\quad#1.}
\makeatother

% Leave date blank
\date{}

% Header and Footer with logo
\usepackage{lastpage,fancyhdr,graphicx}
\usepackage{ragged2e}

\usepackage{color}
\usepackage[dvipsnames]{xcolor}

%\usepackage{epstopdf}
\pagestyle{myheadings}
\pagestyle{fancy}
\fancyhf{}
\lhead{
\includegraphics[width=1.6in]{figures/pone.png}
}
\rhead{Classifying Patents Based on their Semantic Content\vspace{2mm}}
\rfoot{\thepage}
\renewcommand{\footrule}{\hrule height 2pt \vspace{2mm}}
\fancyheadoffset[L]{2.25in}
\fancyfootoffset[L]{2.25in}
\lfoot{\sf Bergeaud, Potiron and Raimbault, 2017}
%\lfoot{}

%% Include all macros below

\newcommand{\lorem}{{\bf LOREM}}
\newcommand{\ipsum}{{\bf IPSUM}}

%%%%%%%%%%%%%%%%%%%%%%
%% User-defined commands
%%%%%%%%%%%%%%%%%%%%%%


%Yoann's commands
\newcommand{\reels}{\mathbb{R}}
\newcommand{\naturels}{\mathbb{N}}
\newcommand{\relatifs}{\mathbb{Z}}
\newcommand{\rat}{\mathbb{Q}}
\newcommand{\complex}{\mathbb{C}}
\newcommand{\esp}{\mathbb{E}}
\newcommand{\proba}{\mathbb{P}}
\newcommand{\var}{\operatorname{Var}}
\newcommand{\cov}{\operatorname{Cov}}
\newcommand{\Tau}{\mathrm{T}}



% writing utilities

% comments and responses
\usepackage{xparse}
\DeclareDocumentCommand{\comment}{m o o o o}
{%
    \textcolor{red}{#1}
    \IfValueT{#2}{\textcolor{blue}{#2}}
    \IfValueT{#3}{\textcolor{ForestGreen}{#3}}
    \IfValueT{#4}{\textcolor{red!50!blue}{#4}}
    \IfValueT{#5}{\textcolor{Aquamarine}{#5}}
}


% todo
\newcommand{\todo}[1]{\textcolor{red!50!blue}{\textbf{\textit{#1}}}}




%% END MACROS SECTION


\begin{document}
\vspace*{0.35in}


\justify









%\end{document}




\section*{S5 Text : Statistical definitions and derivations}
\subsubsection*{Likelihood expression}
We define $\mathcal{F}_i$ the filtration which corresponds to the time $t_i$. With this notation $\mathcal{L}( X | \mathcal{F}_{i-1})$ simply means the likelihood of $X$ conditioned on the past. We consider $\widehat{\theta}^{(z)}$ the MLE\footnote{Apparently, this MLE is a partial MLE, but we will not refer to partial for simplicity.} of $\theta^{(z)}$, where the corresponding log-likelihood of the model can be expressed up to constant terms as
\begin{eqnarray*}
\sum_{i=2}^Z \sum_{l=1}^{C_i} \log \mathcal{L} \big( B_{l,i} | \mathcal{F}_{i-1} \big).
\end{eqnarray*}
Recalling that $B_{l,i}$ are independent of each other and conditioned on the past follow Bernoulli variables 
$$B \Big( \min \Big\{ 1, \frac{N_{i}^{(z)}}{i-1} + \theta^{(z)} \Big\} \Big),$$ 
the log-likelihood of the model can be expressed as 
\begin{eqnarray}
\label{loglik}
\sum_{i=2}^Z \sum_{l=1}^{C_i} B_{l,i} \log \Big( \min \Big\{ 1, \frac{N_{i}^{(z)}}{i-1} + \theta^{(z)} \Big\} \Big) +(1 - B_{l,i}) \log \Big(1 - \min \Big\{ 1, \frac{N_{i}^{(z)}}{i-1} + \theta^{(z)} \Big\} \Big).
\end{eqnarray}
In practice, the user can easily implement the formula (\ref{loglik}) for any $0 \leq \theta^{(z)} \leq 1$, and maximize it over a predefined grid to obtain $\widehat{\theta}^{(z)}$.

\subsubsection*{Standard errors of the estimated values}
Under some assumptions, it is possible to show the asymptotic normality of $\widehat{\theta}^{(z)}$ and to compute the asymptotic variance. For simplicity of exposition, we assume that we restrict to $\theta^{(z)}$ such that we have $\frac{N_{i}^{(z)}}{i-1} + \theta^{(z)} < 1$ for any $i=2, \cdots, Z$. The central limit theorem can be expressed as 
\begin{eqnarray}
\label{variance}
\sqrt{Z \mathbb{E} [ C ]} (\widehat{\theta}^{(z)} - \theta^{(z)}) \overset{\mathcal{L}}{\rightarrow} MN \Big( 0, \int (p + \theta^{(z)})(1 - (p + \theta^{(z)})) d\pi^{(z)} (p) \Big),
\end{eqnarray}
where MN stands for a multinormal distribution and $\pi^{(z)}$ for the asymptotic limit distribution of the quantity $\frac{N_{i}^{(z)}}{i-1} + \theta^{(z)}$. Note that the variance term in (\ref{variance}) is equal to an aggregate version of the Fisher information matrix. The proof of such statement is beyond the scope of this paper. On the basis of (\ref{variance}), we provide a variance estimator as 
$$v^{(z)} = \frac{1}{C_k-1}\sum_{i=2}^{C_k}  \frac{N_{i_k}^{(z)}}{i-1} + \widehat{\theta}^{(z)},$$ 
where $i_k$ is such that the $i_k$th patent corresponds to the $k$th couple. This estimator was used to compute the standard deviation in Table \ref{summary}.
\subsubsection*{Test statistic}
The test statistic used is a mean difference test statistic between $\widehat{\theta}^{(tec)}$ and  $\widehat{\theta}^{(sem)}$, where the formal expression can be found in (\ref{teststat}). We assume independence between both quantities and thus under the null hypothesis, we have that 
$$\widehat{\theta}^{(tec)} - \widehat{\theta}^{(sem)} {\rightarrow} MN(0,V),$$
where $V= \int (p + \theta^{(tec)})(1 - (p + \theta^{(tec)})) d\pi^{(tec)} (p) + \int (p + \theta^{(sem)})(1 - (p + \theta^{(sem)})) d\pi^{(sem)} (p)$ can be estimated by $\widehat{V} = v^{(sem)} + v^{(tec)}$. Then, we obtain that 
\begin{eqnarray}
\label{teststat}
A = \frac{\widehat{\theta}^{(tec)} - \widehat{\theta}^{(sem)}}{\widehat{V}} \approx \mathcal{N} (0,1),
\end{eqnarray}
where $A$ is the mean difference test static.







\end{document}